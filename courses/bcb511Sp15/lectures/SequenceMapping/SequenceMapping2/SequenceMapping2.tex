\documentclass[pdf]{beamer}
\usepackage[latin1]{inputenc}
\usepackage{multirow}
\usetheme{Warsaw} %Warsaw
\usecolortheme{seahorse}
\usepackage[T1]{fontenc}

\usepackage{verbatim}
\begin{document}

\title[Sequence Mapping]{Sequence Mapping/Alignment 2}
\subtitle{BCB 511: Applied Bioinformatics\\}
\author[Matt Settles]{Matt Settles}
\date{\today}


%% Title page
\begin{frame}[plain]
  \titlepage
\end{frame}


%% Outline
\begin{frame}[plain] 
  \frametitle{Outline}
  \tableofcontents
\end{frame}

\section{Alignment/Mapping Applications}

\subsection{Bowtie2}

\begin{frame}
  \frametitle{Bowtie2}
Bowtie2 is an many mapping application developed by Ben Langmead from John Hopkin's University.\\
As with all modern mappers, you need to first build a reference index, then map reads against the index.\\
\begin{footnotesize}
 \begin{verbatim*}
bowtie2-build [options]* <reference\_in> <bt2\_index\_base>\\
bowtie2 [options]* -x <bt2-idx> \{-1 <m1> -2 <m2> | -U <r>\} [-S <sam>]
\end{verbatim*}
\end{footnotesize}
\end{frame}

\begin{frame}
  \frametitle{Local vs Global Mode}
\begin{footnotesize}
\begin{verbatim*}
Global mode forces reads ends to map, no clipping (SOFT or HARD)\\
  For --end-to-end:\\
   --very-fast\\
   --fast\\ 
   --sensitive\\
   --very-sensitive\\
Local mode allows for SOFT and HARD clipping of ends \\ to allow for better alignments\\
  For --local:\\
   --very-fast-local \\
   --fast-local\\
   --sensitive-local\\
   --very-sensitive-local\\
\end{verbatim*}
\end{footnotesize}
\end{frame}
 
\subsection{BWA}

\begin{frame}
  \frametitle{BWA MEM}
  \begin {verbatim*}
  The BWA algorithm was developed by Heng Li from the Broad Institute.\\
   bwa index [options]* <in.fasta>\\
   bwa mem [options]* <idxbase> <in1.fq> [in2.fq]\\
\end {verbatim*}
\end{frame}

\section{SAM/BAM file manipulation}
\subsection{samtools}
\begin{frame}
  \frametitle{samtools}
  http://www.htslib.org
\end{frame}


\subsection{PicardTools}

\begin{frame}
  \frametitle{PicardTools}
  http://broadinstitute.github.io/picard/
\end{frame}

\section{configure\_Mapping.R}

\begin{frame}
  \frametitle{configure\_Mapping.R}
 Like preproc\_experiment, configure\_Mapping.R also automates the process of mapping and some samtools processes (and optionally  read extraction) and assumes the directory structure created by proproc\_experiment and the presence of a samples.txt file.\\
the script is a part of grc/2.0\\
configure\_Mapping.R [options]\\
By default, the script uses BWA MEM\\
When choosing Bowtie2, default is to use global mode, \\- -localmode flag to change to local mode\\
\end{frame}

\begin{frame}
\frametitle{configure\_Mapping.R continued}
Reference:\\
configure\_Mapping.R will index your references for you, or use an index if one already exists.\\
\vspace{0.1in}
Specify a fasta file file to use as a reference, or a mapping\_targets.txt file when mapping agains multiple references.\\
mapping\_targets.txt can be a tab delimited file specifying the name and location of the mapping references. Should have 2 columns
target name and target fasta location, one row for each reference you would like to map each sample to.
\end{frame}

\begin{frame}
\frametitle{configure\_Mapping.R reports}
The script produces 3 types of reports found under the mapping results folder.\\
\begin{enumerate}
\item MappingFlagstats.txt - Tabled of results summering the flagstat report for each sample and target
\item SummarySample2Targets.txt - Report across each target, reporting which target is most like the closest (most mappable reference)
\item summary reads/proportions reports - for each target a report across samples detailing the number of reads and proportion of reads that map to each contig for each sample.
\end{enumerate}

\end{frame}

\end{document}
