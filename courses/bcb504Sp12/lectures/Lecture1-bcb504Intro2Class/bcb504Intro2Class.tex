\documentclass[pdf]{beamer}
\usepackage[latin1]{inputenc}
\usetheme{Warsaw} %Warsaw
\usecolortheme{seahorse}
\title[Course Introduction]{Introduction to Course\\}
\subtitle{BCB 504: Applied Bioinformatics\\}
\author[Matt Settles]{Matt Settles and Matt Pennell}
\institute{University of Idaho\\ Bioinformatics and Computational Biology Program}
\date{Jan 10, 2012}
\begin{document}

\begin{frame}
\titlepage
\end{frame}


\begin{frame}
	\frametitle{Course description and goals}
	\begin{description}
	\item[Description]
	A data driven approach for the computational and statistical understanding and expertise needed to solve bioinformatics problems that you will likely encounter in your research. 
	\item[Goals]
	Following this course the student will be capable of:
	\begin{itemize} 
		\item performing their own data analysis project, 
		\item understanding the technical and statistical tools needed to conduct that analysis
		\item have the computational ability to do the analysis
		\item critically review and implement techniques and methods in publications.
	\end{itemize}
	\end{description}
\end{frame}

\begin{frame}
	\frametitle{Course topics}
	\begin{description}
		\item[DNA Microarrays] Expression, ChIP-Chip, aCGH - 5 lectures
		\item[Phylogenetic Methods] M. Pennell - 4 lectures
		\item[DNA/RNA sequencing] - introductory lecture 
		\begin{itemize}
			\item Sequence Assembly - 3 lectures
			\item Sequence Mapping - 3 lectures
			\item RNA-seq - 3 lectures
			\item Metagenomics - 3 lectures
		\end{itemize}
		\item[GWAS] Genomic Wide Association Studies (aka SNP Chips) - 4 lectures
	\end{description}
\end{frame}

\begin{frame}
	\frametitle{Course format}
	\begin{description}
		\item[Lectures] PDF/Powerpoint style lectures going over the primary topics, analysis objectives, result expectations.
		\item[Workshop] Applying the topics discussed in the lectures on a real dataset.
		\item[Paper discussion] Discussion of methods based techniques within published papers.
	\end{description}
\end{frame}	

\begin{frame}
	\frametitle{Expected capabilities}
	\begin{itemize}
		\item Navigate a Linux environment without trouble (i.e log in to CRC servers, move around directories, create directories, etc.)
		\item Run command line programs and manipulate arguments
		\item Basic familiarity with the R programming language.
	\end{itemize}
\end{frame}
\begin{frame}
	\frametitle{Grading}
	\begin{description}
		\item[Projects (72\%)] For each major section in the class there will be a project associated with it. Each project will be a report on the analysis of public data (or your own data) using the techniques discussed in class. The reports must be written using Latex with embedded R code of the complete analysis. A template will be provided.
		\item[Publication Reviews (28\%)] short 1/2 to full page comments on assigned methods papers. 
	\end{description}
\end{frame}
\begin{frame}
	\frametitle{Course webpage}
	http://www.webpages.uidaho.edu/msettles/courses/bcb504Sp12/index.html
\end{frame}

\end{document}